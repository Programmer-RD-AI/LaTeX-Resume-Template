\documentclass[10pt,letterpaper]{article}
\usepackage[left=0.75in,top=0.6in,right=0.75in,bottom=0.6in]{geometry}
\usepackage{hyperref}
\usepackage{enumitem}
\usepackage{xcolor}
\usepackage{titlesec}
\usepackage{graphicx}

% Set fonts
\usepackage[T1]{fontenc}
\usepackage{libertine}

% Define colors
\definecolor{darkgray}{RGB}{64,64,64}
\definecolor{lightgray}{RGB}{192,192,192}

% Define section formatting
\titleformat{\section}
{\color{darkgray}\normalfont\Large\bfseries}
{}{0em}{}[\titlerule]

% Define subsection formatting
\titleformat{\subsection}
{\color{darkgray}\normalfont\large\bfseries}
{}{0em}{}

% Remove indentation
\setlength{\parindent}{0pt}

% Define a command to create a custom bar
\newcommand{\custombar}[1]{%
    \textcolor{black!30}{\rule{#1\textwidth}{8pt}}%
}

\begin{document}

% Header
\noindent
{\LARGE \textbf{Ranuga Disansa}} \\
\href{mailto:go2ranuga@gmail.com}{go2ranuga@gmail.com} $\vert$ \href{https://www.linkedin.com/in/programmer-rd-ai/}{linkedin.com/in/programmer-rd-ai} $\vert$ \href{https://github.com/Programmer-RD-AI}{github.com/Programmer-RD-AI}

\section*{Summary}
Ranuga Disansa is a creative and analytical software engineering student with a passion for technology and a desire to push the boundaries of what is possible. Proficient in multiple programming languages and familiar with a wide range of software development tools.

\section*{Education}
\textbf{Kingston University} \\
BSc (Hons) Computer Science (Software Engineering) (Sri Lanka ESOFT) Top Up \hfill Jan 2024 - Feb 2025 \\
\textbf{ESOFT Metro Campus} \\
BSc (Hons) Computer Science (Software Engineering) Top Up and BCS Diploma Level \hfill Dec 2020 - Feb 2025 \\
\textbf{Informatics Institute of Technology (IIT Campus)} \\
Foundation Certificate in Higher Education – IT, Artificial Intelligence and Data Science \hfill Sep 2023 - Sep 2028 \\
\textit{Starting BSc (Hons) Computer Science from the University of Westminster in September of 2024}

\section*{Experience}
\textbf{Technical Writer} \hfill Dec 2023 - Present \\
Medium, Sri Lanka \\
- Write articles on various technical topics, including software development, machine learning, and artificial intelligence. (Contributions available on \href{https://medium.com/@Programmer-RD-AI}{Medium Profile}, where articles are independently released)

\section*{Projects}
\textbf{Find-Card} \hfill Feb 2022 - May 2023 \\
- An Object Detection Project using Detectron2 (by @META) \\
- Utilized Detectron2 and Open Images V6 dataset from Google \\
\href{https://github.com/Programmer-RD-AI/Find-Card}{GitHub Link}

\textbf{DataBridgeX} \hfill Month Year - Present \\
- Developed DataBridgeX, a data integration platform that enables seamless communication between different data sources \\
- Implemented features such as data mapping, transformation, and synchronization \\
\href{https://github.com/Programmer-RD-AI/DataBridgeX}{GitHub Link}

\textbf{CivicCircle: AI-driven Solution} \hfill May 2024 - Present \\
- Developed innovative AI solution as part of Colombo AI Conclave 2024 MindStudio Hackathon \\
- Recognized as a Top 10 Finalist among numerous submissions, demonstrating technical proficiency \\
\href{https://github.com/Programmer-RD-AI/CivicCircle}{GitHub Link}

\textbf{Dimensionality-Reduction (DimRed)} \hfill Apr 2024 - Apr 2024 \\
- Navigating the Complexity of High Dimensions: A Comprehensive Guide to Dimensionality Reduction Techniques \\
- Developed DimRed, a comprehensive Python toolkit for advanced dimensionality reduction, integrating with major machine learning libraries and featuring real-time performance monitoring to enhance data analysis and model efficiency \\
\href{https://github.com/Programmer-RD-AI/DimRed}{GitHub Link}

\textbf{NLP Disaster Tweets Classifier} \hfill Month Year - Month Year \\
- Predicted real disaster tweets using natural language processing techniques \\
- Utilized PyTorch and TorchText for model development \\
\href{https://github.com/Programmer-RD-AI/NLP-Disaster-Tweets}{GitHub Link}

\textbf{Digit Recognizer V2} \hfill Jun 2023 - Jun 2023 \\
- Learn computer vision fundamentals with the famous MNIST data \\
- Developed a deep learning model using PyTorch \\
\href{https://github.com/Programmer-RD-AI/Digit-Recognizer-V2}{GitHub Link}

\section*{Skills}
\begin{itemize}[leftmargin=*,label={\color{darkgray}\textbullet}]
    \item \textbf{Programming Languages:} Python, HTML, CSS, JavaScript, TypeScript, SQL
\item \textbf{Web Development:} Flask, Bootstrap
\item \textbf{API Development:} Flask Restful, FastAPI, ExpressJS
\item \textbf{Machine Learning:} PyTorch, Sklearn
\item \textbf{Deep Learning:} PyTorch, Detectron2
\item \textbf{Databases:} MongoDB, SQLite, SQL
\item \textbf{Cloud Providers:} Azure, Firebase, GCP
\item \textbf{Other Tools:} Git, Postman, BeautifulSoup
\item \textbf{No Code:} MindStudio
\end{itemize}

\section*{Honors and Awards}
\textbf{Finalist (Top 10) - Colombo AI Conclave 2024 MindStudio Hackathon} \hfill May 2024 \\
\textbf{3x Kaggle Expert (Discussion, Datasets, Notebooks)} \hfill Jan 2023 \\
\textbf{2022 Young Computer Scientist Challenge Junior Section Award} \hfill Nov 2022 \\
\textbf{1st Runner Up - Hack:AI Hackathon School Category} \hfill Aug 2022 \\
\textbf{2020/2021 APICTA Nominee} \hfill Oct 2021 \\
\textbf{Second Place Winner - YCS 2021 Junior Section} \hfill Jun 2021 \\
\textbf{Junior Champion - Code Fest 2020} \hfill Feb 2021 \\

\section*{Volunteer}

\textbf{Microsoft Open Source Developer (Volunteer)} \hfill May 2024 - Present \\
Microsoft, Science and Technology \\
\textit{Contributed to Microsoft FLAML, addressing issues like missing method arguments and enhancing numerical precision in log outputs. Committed to ongoing development and community engagement.} \\
\href{https://github.com/Programmer-RD-AI/FLAML}{GitHub Link}

\textbf{Industry Outreach Co-Lead} \hfill Mar 2024 - Present \\
IEEE Computer Society Student Branch Chapter of IIT, Science and Technology \\
\textit{Lead efforts to connect our branch with industry, organizing events to bridge academia-industry gap and provide networking opportunities for members.}

\textbf{Industry Outreach Volunteer} \hfill Nov 2023 - Present \\
IEEE Computer Society Student Branch Chapter of IIT, Science and Technology \\
\textit{Recognized as the Industry Outreach (IO) Volunteer of the Month for fostering industry-academia collaboration. Assist in organizing events and coordinating with industry partners.}

\textbf{Google Crowdsource Contributor} \hfill Sep 2023 - Present \\
Google Crowdsource, Science and Technology \\
\textit{Contribute to Google Crowdsource tasks aimed at improving various Google services through user-facing training of different algorithms.} \\
\href{https://crowdsource.app.goo.gl/wk6v}{Crowdsource Link}

\end{document}
